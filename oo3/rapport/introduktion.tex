\documentclass[main.tex]{subfile}
\begin{document}
\section{Introduktion}
I forbindelse med udarbejdelsen af denne rapport har vi arbejdet med kerne moduler til operativsystemet Linux. Arbejdet med modulerne har været opdelt i to hovedogpaver.\\ 

I den første del "\emph{Linux Kernel Modules}" \cite[s.94]{SA:2013} har vi beskæftiget os med udarbejdelsen af et simplet modul for at undersøge hvordan disse skal udformes samt for at undersøge hvordan moduler tilføjes til kernen samt hvordan de fjernes igen efter brug. Her udover har denne del af opgen også fungeret platform for at få indsigt i hvordan en række makroer fungere.\\

Anden halvdel af den obligatoriske opgave er at løse Project 2 - Linux Kernel Module for Listing Tasks \cite[s.156-158]{SA:2013}. Opgaven går ud på at skrive et linux kernemodul, som kan skrive til kernel log bufferen hvilke processer operativsystemet kører ved modulets indlæsningstidpunkt. Delopgaven er yderligere opdelt i 2 dele, og til hver del hører forskellige krav.

\begin{description}
\item[Del 1:] Skab et modul, som itererer igennem alle tasks i systemet. Skriv taskens navn, ID og state til loggen. De er ikke påkrævet at skrive eventuelle barneprocesser til disse tasks.
\item[Del 2:] Skab et modul, som iterer igennem alle tasks i systemet, samt deres barneprocesser, ved brug af en dybde-først algoritme.
\end{description}

Løsningen til første opgave er at finde i filen \texttt{simple.c} og anden opgave er at finde i filerne \texttt{taskPrinter.c} (del 1) og \texttt{taskAndChildPrinter.c} (del 2).
\end{document}